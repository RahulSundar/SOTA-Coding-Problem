\documentclass[12pt]{article}
\usepackage{design_ASC}
\usepackage[utf8]{inputenc}
\usepackage[english]{babel}
\usepackage{listings}
\usepackage{mathtools}
\DeclarePairedDelimiter\ceil{\lceil}{\rceil}
\DeclarePairedDelimiter\floor{\lfloor}{\rfloor}

\usepackage{hyperref}
\hypersetup{
    colorlinks=true,
    linkcolor=blue,
    filecolor=magenta,      
    urlcolor=red,
}

\urlstyle{same}
\setlength\parindent{0pt}

\title{A SOTA Coding Problem\\ \href{https://github.com/Rishit-dagli/SOTA-Coding-Problem}{View on GitHub}}

\author{ \href{https://www.rishit.tech}{Rishit Dagli}\\
\href{https://twitter.com/rishit_dagli}{Twitter} $\vert$ \href{https://github.com/Rishit-dagli}{GitHub} $\vert$
\href{https://medium.com/@rishit.dagli}{Medium} $\vert$
\href{https://www.linkedin.com/in/rishit-dagli}{LinkedIn}
}

\date{\today}

\begin{document}
\setlength{\droptitle}{-5em}    

\maketitle

\section*{Introduction}

I recently came across a State of the Art (SOTA) problem according 
to me. How I define a problem to be really good-

\begin{itemize}
  \item While reading the problem it seems to be very easy
  \item It becomes hard while you try and code it (By logic and not by knowledge of coding)
  \item The optimal solution is just a few lines of code
\end{itemize}

I do not claim that the problem has been made by me in any manner. I
found this question while attempting
\href{https://foobar.withgoogle.com/}{Google FooBar}.

\section*{Dodge the Lasers! (The Problem)}
{\bfseries Someone who you earlier worked for has sent her elite
fighter pilot squadron after you - and they've opened fire!

Fortunately, you know something important about the ships trying to
shoot you down. Back when you were still her assistant, she asked 
you to help program the aiming mechanisms for the ships. They 
undergo rigorous testing procedures, but you were still able to slip
in a subtle bug. The software works as a time step simulation: if it is tracking a target that is accelerating away at $45$ degrees, the
software will consider the targets acceleration to be equal to the
$\sqrt{2}$, adding the calculated result to the targets end velocity
at each timestep. However, thanks to your bug, instead of storing
the result with proper precision, it will be truncated to an integer
before adding the new velocity to your current position.  This means
that instead of having your correct position, the targeting software
will erringly report your position as-
$$\sum_{i=1}^n \floor{i \cdot \sqrt{2}}$$
not far enough off to fail her testing, but enough that it might
just save your life.

If you can quickly calculate the target of the starfighters' laser
beams to know how far off they'll be, you can trick them into
shooting an asteroid, releasing dust, and concealing the rest of
your escape.  Write a function $solution(str_n)$ which, given the string representation of an integer $n$, returns the sum of
$$\floor{1 \cdot \sqrt{2}} + \floor{2 \cdot \sqrt{2}} + \ldots + \floor{n \cdot \sqrt{2}}$$
as a string. That is, for every number $i$ in the range $1$ to $n$,
it adds up all of the integer portions of $i \cdot \sqrt{2}$.

For example, if $str_n$ was $"5"$, the solution would be calculated 
as-

$floor(1*sqrt(2)) +$

$floor(2*sqrt(2)) +$

$floor(3*sqrt(2)) +$

$floor(4*sqrt(2)) +$

$floor(5*sqrt(2))$

$= 1+2+4+5+7 = 19$

so the function would return $"19"$
}

\subsection*{Twist}

$n$ can have values upto $10^100$ which means it can have numbers 
with 101 digits! Try solving this problem yourself first before
heading to the solution.

PS: If it was as easy as a for loop I would not have written this in
the first place

\subsection*{Hints}

This seems like a very easy problem. If you remove the story line, 
all you need to calculate is-

$$\sum_{i=1}^n \floor{i \cdot \sqrt{2}}$$

The only twist is you cant use recursion for each element or use a
for loop as the numbers are too big. You need to try and simplify
the expression.

\section*{Solution}

Just some convention-

This type of sequence is called a  Beatty sequence. Beatty sequence
is the sequence of integers found by taking the floor of the
positive multiples of a positive irrational number. However, you do
not require any knowledge of it to solve the problem. \\

Let's define a function $\beta _r ^{(i)} = \floor{i \cdot r}$ for 
some irrational number and

$$S(r,n) = \sum_{i=1}^n \beta _r ^{(i)}$$

if $r \geq 2$ we say $p = r - 1$ and we can also say-

\begin{align*}
S(p,n)&=S(s,n) + \sum_{i=1}^n \cdot i \\
&=S(s,n) + \frac{n(n+1)}{2}
\end{align*}

Also, if $1 < r < 2$ and $\frac{1}{p} + \frac{1}{r} = 1$ then 
sequences $\beta_r$ and $\beta_p$ for $n \geq 1$ partition
$\mathbb{N}$

So,

$$S(r, n) + S(s, \lfloor \frac{\mathcal{B}_r^{(n)}}{s} \rfloor) =
\sum_i^{\mathcal{B}_r^{(n)}} i =
\frac{\mathcal{B}_r^{(n)}(\mathcal{B}_r^{(n)} + 1)}{2}$$

And also

\begin{align*}
\lfloor \frac{\mathcal{B}_r^{(n)}}{s} \rfloor &= \lfloor
\mathcal{B}_r^{(n)}(1 - \frac{1}{r}) \rfloor \\
&= \mathcal{B}_r^{(n)} - \lceil \frac{\mathcal{B}_r^{(n)}}{r} \rceil
\\ &= \mathcal{B}_r^{(n)} - n
\end{align*}

Then, letting $n^{\prime} = \lfloor (r - 1)n \rfloor =
\mathcal{B}_{r-1}^{n}$
we have 
$$S(r,n) = \frac{\mathcal{B}_r^{(n)}(\mathcal{B}_r^{(n)} + 1)}{2} - S(p, n^{\prime}) = \frac{(n + n^{\prime})(n + n^{\prime} + 1)}{2} - S(p, n^{\prime})$$

Back to the problem, we have $p = \sqrt{2}$, so we start with $p = 2
+ \sqrt{2}$. We can get a recurrence formula.
Let $n^{\prime} = \lfloor (\sqrt{2} - 1)n \rfloor$

\begin{align*}
S(\sqrt{2}, n) 
&= \frac{(n + n^{\prime})(n + n^{\prime} + 1)}{2} - S(2 + \sqrt{2},
n^{\prime})\\
&= \frac{(n + n^{\prime})(n + n^{\prime} + 1)}{2} -
(n^{\prime}(n^{\prime} + 1) - S(\sqrt{2}, n^{\prime}))\\
&= nn^{\prime} + \frac{n(n+1)}{2} - \frac{n^{\prime}(n^{\prime} +
1)}{2} - S(\sqrt{2}, n^{\prime})
\end{align*}

And we solved the problem, got an optimal way to solve it :)

\section*{In Code}

Let us now take a look at implementing this in code. We will just
convert all what we just talked about to code.

\begin{lstlisting}[language=Python]
minus_sqrt2 = 414213562373095048801688724209698078569671875376948 0731766797379907324784621070388503875343276415727
def n1(n):
    return (minus_sqrt2 * n) // (10 ** 100)
def s(n):
    if n == 1:
        return 1
    if n < 1:
        return 0
    return n * n1(n) + n * (n + 1) / 2 - n1(n) * (n1(n) + 1) / 2 - s(n1(n))
def solution(n):
    return str(s(int(n)))
\end{lstlisting}

In case you aren't able to view the rendered code block, navigate 
down to "Try out yourself".

You can see it is just 11 lines of code to tackle this problem! I 
think this is pretty cool, as just your logic was what counted here
and not knowledge of a programming language.

\subsection*{Try out for yourself}

You can try out this code for yourself using these links-
\begin{itemize}
    \item \href{https://github.com/Rishit-dagli/SOTA-Coding-Problem/blob/master/A_SOTA_Coding_problem_solution.ipynb}{Colab environment} 
    - Requires a Google Account (Recommended)
    \item \href{https://repl.it/@RishitDagli/A-SOTA-Coding-Problem}{Repl.it environment}
\end{itemize}

\section*{About Me}

Hi everyone I am Rishit Dagli. This was my first time writing a blog
on a specific problem which I feel could help others develop their
logic. If you like this one and would love to see more of these kind
from me, consider 
\href{https://github.com/Rishit-dagli/SOTA-Coding-Problem}{starring 
the repo}

If you want to ask me some questions, report any mistake, suggest
improvements, give feedback you are free to do so by emailing me at-

\begin{itemize}
    \item rishit.dagli@gmail.com
    \item hello@rishit.tech
\end{itemize}

\end{document}